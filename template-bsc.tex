%% Template para dissertação/tese na classe UFBAthesis
%% versao 1.0
%% (c) 2005 Paulo G. S. Fonseca
%% (c) 2012 Antonio Terceiro
%% (c) 2014 Christina von Flach
%% www.dcc.ufba.br/~flach/ufbathesis

%% Carrega a classe ufbathesis
%% Opcoes: * Idiomas
%%           pt   - portugues (padrao)
%%           en   - ingles
%%         * Tipo do Texto
%%           bsc  - para monografias de graduacao
%%           msc  - para dissertacoes de mestrado (padrao)
%%           qual - exame de qualificacao de mestrado
%%           prop - exame de qualificacao de doutorado
%%           phd  - para teses de doutorado
%%         * Mídia
%%           scr  - para versão eletrônica (PDF) / consulte o guia do usuario
%%         * Estilo
%%           classic - estilo original a la TAOCP (deprecated)
%%           std     - novo estilo a la CUP (padrao)
%%         * Paginacao
%%           oneside - para impressao em face unica
%%           twoside - para impressao em frente e verso (padrao)
\documentclass[bsc, classic, a4paper]{ufbathesis}

%% Preambulo:
\usepackage[utf8]{inputenc}
\usepackage{graphicx}
\usepackage{lipsum}

% Universidade
\university{UNIVERSIDADE FEDERAL DA BAHIA}

% Endereco (cidade)
\address{Salvador}

% Instituto ou Centro Academico
\institute{INSTITUTO DE MATEMÁTICA}

% Nome da biblioteca - usado na ficha catalografica
\library{BIBLIOTECA REITOR MAC\^{E}DO COSTA}

% Programa de pos-graduacao
\program{Programa de Graduação em Ciência da Computação}

% rea de titulacao
\majorfield{CI\^{E}NCIA DA COMPUTA\c{C}\~{A}O}

% Titulo da dissertacao
\title{Uma abordagem fuzzy híbrida para organização de documentos, utilizando os algoritmos de agrupamento possibilístico e fuzzy c means }

% Data da defesa
% e.g. \date{19 de fevereiro de 2003}
\date{1 de junho de 2016}

% Autor
% e.g. \author{Jose da Silva}
\author{Nilton Vasques Carvalho Junior}

% Orientador(a)
% Opcao: [f] - para orientador do sexo feminino
% e.g. \adviser[f]{Profa. Dra. Maria Santos}
\adviser[f]{Profa. Dra. Tatiane Nogueira Rios}

% Orientador(a)
% Opcao: [f] - para orientador do sexo feminino
% e.g. \coadviser{Prof. Dr. Pedro Pedreira}
% Comente se nao se aplicar
%\coadviser{NOME DO(DA) CO-ORIENTADOR(A)}

%% Inicio do documento
\begin{document}

\pgcompfrontpage{}

%% Parte pre-textual
\frontmatter

\pgcomppresentationpage

% Ficha catalografica
\authorcitationname{Carvalho, Nilton Vasques Jr.} % e.g. Terceiro, Antonio Soares de Azevedo
\advisercitationname{Rios, Tatiane Nogueira} % e.g. Chavez, Christina von Flach Garcia
\catalogtype{Monografia (Graduação)} % e.g. ``Tese (Doutorado)''
\catalogtopics{``1. Fuzzy C Means. 2. Organização de documents. 3. Lógica Fuzzy. 4. Mineração de dados.''} % e.g. ``1. Complexidade Estrutural. 2. Engenharia de Software''
\catalogcdd{NUMERO CDD} % e.g. ``CDD 20.ed. XXX.YY'' (esse número vai lhe ser dado pela biblioteca)
\catalogingsheet

% Termo de aprovacaoo
\approvalsheet{Salvador, DIA de MES de ANO}{
   \comittemember{Profa. Dra. Tatiane Nogueira Rios}{Universidade Federal da Bahia}
   \comittemember{Prof. Dr. Professor 2}{Universidade 123}
   \comittemember{Profa. Dra. Professora 3}{Universidade ABC}
}

% Dedicatoria
% Comente para ocultar
\begin{dedicatory}
Coloque sua DEDICATORIA AQUI.
\end{dedicatory}

% Agradecimentos
\acknowledgements
Coloque seus AGRADECIMENTOS AQUI.

% Epigrafe
%  \begin{epigraph}[Tarde, 1919]{Olavo Bilac}
%  Ultima flor do Lacio, inculta e bela,\\
%  Es, a um tempo, esplendor e sepultura;\\
%  Ouro nativo, que, na ganga impura,\\
%  A bruta mina entre os cascalhos vela.
%  \end{epigraph}
\begin{epigraph}[1687]{Isaac Newton}
  O que sabemos é uma gota, o que ignoramos é um oceano.
\end{epigraph}

% Resumo em Portugues
\resumo
%A new powerful and flexible organization of documents can be obtained by mixing fuzzy and possibilistic clustering, in which documents can belong to more than one cluster simultaneously with different compatibility degrees with a particular topic. The topics are represented by clusters and the clusters are identified by one or more descriptors extracted by a proposed method. We aim to investigate whether the descriptors extracted after fuzzy and possibilistic clustering improves the flexible organization of documents. Experiments were carried using a collection of documents and we evaluated the descriptors ability to capture the essential information of the used collection. The results prove that the fuzzy possibilistic clusters descriptors extraction is effective and can improve the flexible organization of documents.

Diante da grande quantidade de informações geradas e armazenadas pela humanidade na atualidade, 
vários métodos foram propostos visando processar esses dados. Dentre esses dados, temos uma imensa
quantidade de dados textuais, que por sua vez são não estruturados. Com isso é notória a importância,
de organizar de maneira automatizada, esses documentos pelos assuntos ao qual se tratam. Em particular temos um conjunto de técnicas pertencentes ao campo de estudo da mineração de textos, que visam realizar a tarefa de extrair informações relevantes de documentos textuais. Esta tarefa de análise e extração de informações é 
comumente segmentada nas tarefas de coleta, pré-processamento dos documentos, agrupamento dos dados
e por fim a extração de descritores dos grupos obtidos na etapa de agrupamento. Os métodos de agrupamento podem ser separados então pela lógica matemática utilizada, que pode ser a lógica clássica ou a lógica fuzzy. Na lógica clássica, após o agrupamento, cada documento só poderá pertencer a um grupo, enquanto na lógica fuzzy, a pertinência do documento será distribuída entre os grupos. 
Se analisarmos a diversidade de conteúdo em documentos textuais, é trivial notar que frequentemente
um texto aborda um ou mais temas. Com isso é evidente a necessidade de desenvolver-se técnicas para
organizar de maneira flexível os documentos. Percebe-se então, que os métodos de agrupamento fuzzy,
se mostram coerentes com a realidade da estrutura dos documentos textuais.
% FALAR SOBRE OS MÈTODOS FUZZY UTILIZADOS, FCM, PFCM, PCM, HFCM, HPCM
% FALAR SOBRE A EXTRAÇÂO DE DESCRITORES
% FALAR SOBRE A MISTURA REALIZADA NA EXTRAÇÃO DE DESCRITORES

% Palavras-chave do resumo em Portugues
\begin{keywords}
agrupamento fuzzy, agrupamento possibilístico, organização flexível de documentos, 
mineração de textos
\end{keywords}

% Resumo em Ingles
\abstract
A new powerful and flexible organization of documents can be obtained by mixing fuzzy and possibilistic clustering, in which documents can belong to more than one cluster simultaneously with different compatibility degrees with a particular topic. The topics are represented by clusters and the clusters are identified by one or more descriptors extracted by a proposed method. We aim to investigate whether the descriptors extracted after fuzzy and possibilistic clustering improves the flexible organization of documents. Experiments were carried using a collection of documents and we evaluated the descriptors ability to capture the essential information of the used collection. The results prove that the fuzzy possibilistic clusters descriptors extraction is effective and can improve the flexible organization of documents.

% Palavras-chave do resumo em Ingles
\begin{keywords}
fuzzy clustering, possibilistic clustering, flexible organization, documents, text mining
\end{keywords}

% Sumario / Indice
% Comente para ocultar
\tableofcontents

% Lista de figuras
% Comente para ocultar
\listoffigures

% Lista de tabelas
% Comente para ocultar
\listoftables

%% Parte textual
\mainmatter

% Eh aconselhavel criar cada capitulo em um arquivo separado, digamos
% "capitulo1.tex", "capitulo2.tex", ... "capituloN.tex" e depois
% inclui-los com:
% \include{capitulo1}
% \include{capitulo2}
% ...
% \include{capituloN}
%
% Importante: Use \xchapter ao inves de \chapter, se quiser colocar texto antes do inicio do capitulo.

\xchapter{Introdu\c{c}\~{a}o}{Este eh o primeiro cap\'{\i}tulo, onde eu conto toda a historia deste trabalho, o problema, a solu\c{c}\~{a}o, etc.}

\lipsum

\xchapter{Revis\~{a}o Bibliogr\'{a}fica}{Neste cap\'{\i}tulo eu apresento todo o material que eu estudei durante a elabora\c{c}\~{a}o do trabalho.}

\lipsum

Livro \cite{demeyer2008} e  livro \cite{raymond1999}.

\chapter{Exemplos}

Figura \ref{default-regular} e tabela \ref{default-table}.
\begin{figure}[htbp]
\begin{center}
  \includegraphics[scale=0.5]{ufba.eps}[0.5]
\caption{Figura UFBA}
\label{default-regular}
\end{center}
\end{figure}

\begin{table}[htbp]
\caption{Tabela Exemplo}
\begin{center}
\begin{tabular}{|c|c|} 
\hline
elemento 11 & elemento 12 \\ \hline
elemento 21 & elemento 22 \\ \hline
elemento 31 & elemento 32 \\
\hline
\end{tabular}
\end{center}
\label{default-table}
\end{table}%

%% Parte pos-textual
\backmatter

% Apendices
% Comente se naoo houver apendices
\appendix

% Eh aconselhavel criar cada apendice em um arquivo separado, digamos
% "apendice1.tex", "apendice.tex", ... "apendiceM.tex" e depois
% inclui--los com:
% \include{apendice1}
% \include{apendice2}
% ...
% \include{apendiceM}

% Bibliografia
% É aconselhável utilizar o BibTeX a partir de um arquivo, digamos "biblio.bib".
% Para ajuda na criação do arquivo .bib e utilização do BibTeX, recorra ao
% BibTeXpress em www.cin.ufpe.br/~paguso/bibtexpress
\bibliographystyle{abntex2-alf}
\bibliography{biblio}

%% Fim do documento
\end{document}
%------------------------------------------------------------------------------------------%
