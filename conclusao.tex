Na pesquisa apresentada nesta monografia, foi investigado e apresentado o processo de organizar
uma coleção de documentos textuais de maneira flexível. Esse processo conforme ficou evidenciado
durante esse texto, é um procedimento de aquisição de conhecimento bastante intensivo, no qual é
necessário se aplicar um diversificado conjunto de técnicas para mitigar os diversos desafios que
circundam esta árdua tarefa. Sendo que boa parte das técnicas que são empregadas, derivam da
mineração de dados e consequentemente também da mineração de textos. Foi também pontuado ao longo do
texto, os desafios existentes nesta área de pesquisa, como por exemplo, os impactos negativos da
elevada dimensionalidade da matriz documentos x termos, que é naturalmente bastante esparsa em
coleções textuais. Onde por sua vez, dificultam bastante a tarefa de se calcular a similaridade
entre dois documentos quaisquer. Outro ponto crucial, está no processo de agrupamento, onde se é
esperado que os grupos resultantes, consigam capturar a estrutura natural das coleções, de maneira
que os grupos resultantes possuam relevância para os usuários, e consequentemente cumprir o papel
de aquisição e descoberta de conhecimento. Coleções textuais, podem também conter documentos
ruidosos, que destoam do restante da coleção, portanto se é esperado que o processo de organização
flexível, não seja prejudicado pela presença desses documentos. E com o aumento massivo da
quantidade de dados produzidos pela humanidade, se faz também necessário que todo o processo, seja
capaz de adequar para coleções com grande volumes de dados.

Todo esse contexto, apresentado se mostra inviável de ser abordado de maneira completa em uma única
pesquisa, por isso, as investigações conduzidas nessa monografia foram focadas no aumento da
robustez do processo, reduzindo-se os impactos dos dados ruidosos ao se utilizar uma estratégia
híbrida com o algoritmo PFCM. Portanto a hipótese formulada e verificada nesta monografia foi:

\begin{quote}
\textit{A utilização de uma estratégia híbrida de agrupamento e extração de descritores, entre os 
  graus de pertinência e tipicidade providos pelo método de agrupamento PFCM, permitem o aumento da
    robustez e resiliência contra ruídos na organização flexível de documentos, aumentando assim a
    relevância dos grupos obtidos.}
\end{quote}

Portanto, com base na exploração das estratégias existentes na literatura para aprimoramento do
processo de organização flexível de documentos, e da avaliação da hipótese formulada, o objetivo
desta monografia é definido como segue:

\begin{quote}
\textit{Conduzir uma investigação em torno dos métodos de agrupamento FCM, PCM e PFCM, para
se compreender e interpretar corretamente as peculiaridades de se extrair descritores em um
agrupamento híbrido.}
\end{quote}
